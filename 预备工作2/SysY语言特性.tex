% -*- coding: utf-8 -*-
%-------------------------designed by zcf--------------
\documentclass[UTF8,a4paper,10pt]{ctexart}
\usepackage[left=3.17cm, right=3.17cm, top=2.74cm, bottom=2.74cm]{geometry}
\usepackage{amsmath}
\usepackage{graphicx,subfig}
\usepackage{float}
\usepackage{cite}
\usepackage{caption}
\usepackage{enumerate}
\usepackage{booktabs} %表格
\usepackage{multirow}
\newcommand{\tabincell}[2]{\begin{tabular}{@{}#1@{}}#2\end{tabular}}  %表格强制换行
%-------------------------字体设置--------------
% \usepackage{times} 
\usepackage{ctex}
\setCJKmainfont[ItalicFont=Noto Sans CJK SC Bold, BoldFont=Noto Serif CJK SC Black]{Noto Serif CJK SC}
\newcommand{\yihao}{\fontsize{26pt}{36pt}\selectfont}           % 一号, 1.4 倍行距
\newcommand{\erhao}{\fontsize{22pt}{28pt}\selectfont}          % 二号, 1.25倍行距
\newcommand{\xiaoer}{\fontsize{18pt}{18pt}\selectfont}          % 小二, 单倍行距
\newcommand{\sanhao}{\fontsize{16pt}{24pt}\selectfont}  %三号字
\newcommand{\xiaosan}{\fontsize{15pt}{22pt}\selectfont}        % 小三, 1.5倍行距
\newcommand{\sihao}{\fontsize{14pt}{21pt}\selectfont}            % 四号, 1.5 倍行距
\newcommand{\banxiaosi}{\fontsize{13pt}{19.5pt}\selectfont}    % 半小四, 1.5倍行距
\newcommand{\xiaosi}{\fontsize{12pt}{18pt}\selectfont}            % 小四, 1.5倍行距
\newcommand{\dawuhao}{\fontsize{11pt}{11pt}\selectfont}       % 大五号, 单倍行距
\newcommand{\wuhao}{\fontsize{10.5pt}{15.75pt}\selectfont}    % 五号, 单倍行距
%-------------------------章节名----------------
\usepackage{ctexcap} 
\CTEXsetup[name={,、},number={ \chinese{section}}]{section}
\CTEXsetup[name={(,)},number={\chinese{subsection}}]{subsection}
\CTEXsetup[name={,.},number={\arabic{subsubsection}}]{subsubsection}
%-------------------------页眉页脚--------------
\usepackage{fancyhdr}
\pagestyle{fancy}
\lhead{\kaishu \leftmark}
% \chead{}
\rhead{\kaishu 编译原理实验报告}%加粗\bfseries 
\lfoot{}
\cfoot{\thepage}
\rfoot{}
\renewcommand{\headrulewidth}{0.1pt}  
\renewcommand{\footrulewidth}{0pt}%去掉横线
\newcommand{\HRule}{\rule{\linewidth}{0.5mm}}%标题横线
\newcommand{\HRulegrossa}{\rule{\linewidth}{1.2mm}}
%-----------------------伪代码------------------
\usepackage{algorithm}  
\usepackage{algorithmicx}  
\usepackage{algpseudocode}  
\floatname{algorithm}{Algorithm}  
\renewcommand{\algorithmicrequire}{\textbf{Input:}}  
\renewcommand{\algorithmicensure}{\textbf{Output:}} 
\usepackage{lipsum}  
\makeatletter
\newenvironment{breakablealgorithm}
  {% \begin{breakablealgorithm}
  \begin{center}
     \refstepcounter{algorithm}% New algorithm
     \hrule height.8pt depth0pt \kern2pt% \@fs@pre for \@fs@ruled
     \renewcommand{\caption}[2][\relax]{% Make a new \caption
      {\raggedright\textbf{\ALG@name~\thealgorithm} ##2\par}%
      \ifx\relax##1\relax % #1 is \relax
         \addcontentsline{loa}{algorithm}{\protect\numberline{\thealgorithm}##2}%
      \else % #1 is not \relax
         \addcontentsline{loa}{algorithm}{\protect\numberline{\thealgorithm}##1}%
      \fi
      \kern2pt\hrule\kern2pt
     }
  }{% \end{breakablealgorithm}
     \kern2pt\hrule\relax% \@fs@post for \@fs@ruled
  \end{center}
  }
\makeatother
%------------------------代码-------------------
\usepackage{xcolor} 
\usepackage{listings} 
\lstset{ 
breaklines,%自动换行
basicstyle=\small,
escapeinside=``,
keywordstyle=\color{ blue!70} \bfseries,
commentstyle=\color{red!50!green!50!blue!50},% 
stringstyle=\ttfamily,% 
extendedchars=false,% 
linewidth=\textwidth,% 
numbers=left,% 
numberstyle=\tiny \color{blue!50},% 
frame=trbl% 
rulesepcolor= \color{ red!20!green!20!blue!20} 
}
%------------超链接----------
\usepackage[colorlinks,linkcolor=black,anchorcolor=blue]{hyperref}
%------------------------TODO-------------------
\usepackage{enumitem,amssymb}
\newlist{todolist}{itemize}{2}
\setlist[todolist]{label=$\square$}
% for check symbol 
\usepackage{pifont}
\newcommand{\cmark}{\ding{51}}%
\newcommand{\xmark}{\ding{55}}%
\newcommand{\done}{\rlap{$\square$}{\raisebox{2pt}{\large\hspace{1pt}\cmark}}\hspace{-2.5pt}}
\newcommand{\wontfix}{\rlap{$\square$}{\large\hspace{1pt}\xmark}}
%------------------------水印-------------------
\usepackage{tikz}
\usepackage{xcolor}
\usepackage{eso-pic}

\newcommand{\watermark}[3]{\AddToShipoutPictureBG{
\parbox[b][\paperheight]{\paperwidth}{
\vfill%
\centering%
\tikz[remember picture, overlay]%
  \node [rotate = #1, scale = #2] at (current page.center)%
    {\textcolor{gray!80!cyan!30!magenta!30}{#3}};
\vfill}}}



%———————————————————————————————————————————正文———————————————————————————————————————————————
%----------------------------------------------
\begin{document}
\begin{titlepage}
    \begin{center}
    \includegraphics[width=0.8\textwidth]{NKU.png}\\[1cm]    
    \textsc{\Huge \kaishu{\textbf{南\ \ \ \ \ \ 开\ \ \ \ \ \ 大\ \ \ \ \ \ 学}} }\\[0.9cm]
    \textsc{\huge \kaishu{\textbf{网\ \ 络\ \ 空\ \ 间\ \ 安\ \ 全\ \ 学\ \ 院}}}\\[0.5cm]
    \textsc{\Large \textbf{编译原理实验报告}}\\[0.8cm]
    \HRule \\[0.9cm]
    { \LARGE \bfseries 实现SysY编译器}\\[0.4cm]
    \HRule \\[2.0cm]
    \centering
    \textsc{\LARGE \kaishu{周延霖\ 2013921}}\\[0.5cm]
    \textsc{\LARGE \kaishu{年级\ :\ 2020级}}\\[0.5cm]
    \textsc{\LARGE \kaishu{专业\ :\ 信息安全}}\\[0.5cm]
    \textsc{\LARGE \kaishu{指导教师\ :\ 王刚}}\\[0.5cm]
    \vfill
    {\Large \today}
    \end{center}
\end{titlepage}
%-------------摘------要--------------
\newpage
\thispagestyle{empty}
\renewcommand{\abstractname}{\kaishu \sihao \textbf{摘要}}
    \begin{abstract}
        此次实验是对编译器和整个编译过程的了解,在学习了高级语言以及计算机组成原理这些基础课之后,也应该了解一些关于执行文件方式方面的知识,在本次试验中也对自己将要实现的SysY语言特性进行LLVM IR程序的编写
        \noindent  %顶格
        \textbf{\\\ 关键字:}SysY语言、形式化定义、arm汇编\textbf{} \\\ \\\
    \end{abstract}
%----------------------------------------------------------------
\tableofcontents
%----------------------------------------------------------------
\newpage
\watermark{60}{10}{NKU}
\setcounter{page}{1}




%----------------------------------------------------------------
\section{SysY语言特性和形式化定义}
\subsection{SysY语言特性}
Sysy语言是C语言的一个子集,因此继承了C语言的语法定义和特性。由函数、常变量声明、语句、表达式等多种元素构成。接下来,我们将对这种语言的每个部分的特性进行具体分析。
\subsubsection{关键字}C++常用关键字如表\ref{fig:1}所示,每一个关键字在上下文无关文法中都会看作一个终结符,即语法树的叶结点。本实验将选取其中的一部分作为子集,构造SysY语言。
\begin{table}[!htbp]
  \centering
  \begin{tabular}{ccccccccccc}
  \toprule  
  类型& 关键字&\\
  \midrule
  数据类型相关&$int,bool,true,false,char,wchar\_t,int,double,float,short,$&\\
  & $long,signed,unsigned$&\\
  控制语句相关&$switch, case, default, do, for, while, if, else,$&\\
  & $break, continue, goto$&\\
  定义、初始化相关& $const, voltile, enum, export, extern, public, protected, private,$&\\ 
   & $template, static, struct, class, union, mutable, virtual$&\\
  系统操作相关& $catch, throw, try, new, delete, friend, inline, $&\\
  & $operator, retuster, typename$&\\
  命名相关& $using, namespace, typeof$&\\
  函数和返回值相关& $void, return, sizeof, typied$&\\
  其他& $this, asm, \*\_cast$&\\
  \bottomrule
  \end{tabular}
  \caption{C++关键字}
  \label{fig:1}
\end{table}
\par
其中,由于功能相对简单,故为了满足最基本要求,我们为SysY语言选取的保留字如表\ref{fig:2}所示:
\begin{table}[!htbp]
  \centering
  \begin{tabular}{ccccccccccc}
  \toprule  
  类型& 关键字&\\
  \midrule
  数据类型相关&$int,char$&\\
  控制语句相关&$while, if, else, break, continue$&\\
  定义、初始化相关& $const$&\\ 
  系统操作相关& $new, delete$&\\
  命名相关& $using, namespace$&\\
  函数和返回值相关& $void, return$&\\
  \bottomrule
  \end{tabular}
  \caption{SysY语言关键字}
  \label{fig:2}
\end{table}
\subsubsection{变量}
\textbf{C语言中规定,将一些程序运行中可变的值称之为变量,与常量相对。}在程序运行期间,随时可能产生一些临时数据,应用程序会将这些数据保存在一些内存单元中,每个内存单元都用一个标识符来标识。这些内存单元我们称之为变量,定义的标识符就是变量名,内存单元中存储的数据就是变量的值。\cite{bl}变量可以作左值,常量则只能作为右值。变量除了与常量相同的整型类型、实型类型、字符类型这三个基本类型之外,还有构造类型、指针类型、空类型。
\paragraph{三种基本类型}
\subparagraph{整型变量}
  整型常量为整数类型$int$的数据。可分别如下表示为八进制、十进制、十六进制
    \begin{itemize}
    \item 十进制整型变量:0, 123, -1
    \item 八进制整型变量:0123, -01
    \item 十六进制整型变量:0x123,-0x88
    \end{itemize}
  \subparagraph{实型变量} 实型变量是实际中的小数,又称为浮点型变量。按照精度可以分为单精度浮点数(float)和双精度浮点数(double)。浮点数的表示有三种方式如下:
    \begin{itemize}
    \item 指明精度的表示:以f结尾为单精度浮点数,如:2.3f;以d结尾为双精度浮点数,如:3.6d
    \item 不加任何后缀的表示:11.1,5.5
    \item 指数形式的表示:5.022e+23f,0f
    \end{itemize}
  \subparagraph{字符变量} $char$用于表示一个字符,表示形式为$'$需要表示的字符常变量$'$。其中,所表示的内容可以是英文字母、数字、标点符号以及由转义序列来表示的特殊字符。如$'$a$'$\ \  $'$3$'$\ \  $'$,$'$\ \  $'$$\backslash$n$'$。
\paragraph{变量的定义} 用于为变量分配存储空间,还可为变量指定初始值。程序中,变量有且仅有一个定义。\cite{m}变量有三个基本要素:变量名,代表变量的符号;变量的数据类型,每一个变量都应具有一种数据类型且内存中占据一定的储存空间;变量的值,变量对应的存贮空间中所存放的内容。变量的定义可以以如下的形式:
\begin{lstlisting}[language = c++]
type variable_list
\end{lstlisting}
\textbf{在我们定义的SysY语言中,将支持整型变量(十进制)、字符型变量以及行主存储的整型一维数组类型。}

\subsubsection{常量}
\textbf{C语言中规定,将一些不可变的值称之为常量。}常量可以分为整型常量、实型常量、字符常量这三种常量,其形式与整型变量、实型变量、字符变量基本相同,只不过在声明时必须初始化且在程序中不可以改变其值。\textbf{在我们定义的SysY语言中,将定义整型常量(十进制)和字符型常量。}

\subsubsection{运算符和表达式}
在C语言中,运算符分为算术运算符、关系运算符、逻辑运算符、位运算符、赋值运算符、杂项运算符六大类。其中,我们所设计的SysY语言将定义以下运算符:
\begin{table}[!htbp]
  \centering
  \begin{tabular}{ccccccccccc}
  \toprule  
  运算符& 描述&\\
  \midrule
  +& 把两个操作数相加&\\
  -& 从第一个操作数中减去第二个操作数&\\  
  *& 把两个操作数相乘&\\
  /& 分子除以分母&\\
  \%& 取模运算符,整除后的余数&\\
  \bottomrule
  \end{tabular}
  \caption{算术运算符}
\end{table}
\begin{table}[!htbp]
  \centering
  \begin{tabular}{ccccccccccc}
  \toprule  
  运算符& 描述&\\
  \midrule
  ==&  检查两个操作数的值是否相等,如果相等则条件为真。& \\
  !=&   检查两个操作数的值是否相等,如果不相等则条件为真。& \\
  >&  检查左操作数的值是否大于右操作数的值,如果是则条件为真。& \\
  <&  检查左操作数的值是否小于右操作数的值,如果是则条件为真。&\\
  >=&   检查左操作数的值是否大于或等于右操作数的值,如果是则条件为真。& \\
  <=&   检查左操作数的值是否小于或等于右操作数的值,如果是则条件为真。& \\
  \bottomrule
  \end{tabular}
  \caption{关系运算符}
\end{table}
\begin{table}[!htbp]
  \centering
  \begin{tabular}{ccccccccccc}
  \toprule  
  运算符& 描述&\\
  \midrule
  \&\& &  称为逻辑与运算符。如果两个操作数都非零,则条件为真。&\\
  ||&  称为逻辑或运算符。如果两个操作数中有任意一个非零,则条件为真。&\\
  !& 称为逻辑非运算符。用来逆转操作数的逻辑状态。&\\
  \bottomrule
  \end{tabular}
  \caption{逻辑运算符}
\end{table}
\begin{table}[!htbp]
  \centering
  \begin{tabular}{ccccccccccc}
  \toprule  
  运算符& 描述&\\
  \midrule
  =& 简单的赋值运算符,把右边操作数的值赋给左边操作数。&\\
  \bottomrule
  \end{tabular}
  \caption{赋值运算符}
\end{table}
\begin{table}[!htbp]
  \centering
  \begin{tabular}{ccccccccccc}
  \toprule  
  运算符& 描述&\\
  \midrule
  \& & 返回变量的地址。&\\
  * & 指向一个变量。&\\
  \bottomrule
  \end{tabular}
  \caption{复杂运算符}
\end{table}
\paragraph{表达式}由运算分量和运算符按一定规则组成。运算分量是运算符操作的对象,通常是各种类型的数据。运算符指明表达式的类型;表达式的运算结果是一个值——表达式的值。出现在赋值运算符左边的分量为左值,代表着一个可以存放数据的存储空间;左值只能是变量,不能是常量或表达式,因为只有变量才可以带表存放数据的存储空间。出现在赋值运算符右边的分量为右值,右值没有特殊要求。
\paragraph{运算符优先级}运算符中优先级确定了表达式中项的组合,这会极大地影响表达式的计算过程以及结果。运算的优先顺序为:括号优先运算$\rightarrow$优先级高的运算符优先运算$\rightarrow$优先级相同的运算参照运算符结合性依次进行。
当表达式包含多个同级运算符时,运算的先后次序分为左结合规则和右结合规则。其中左结合规则是从左向右依次计算,包括的运算符有双目的算术运算符、关系运算符、逻辑运算符、位运算符、逗号运算符;右结合规则是从右向左依次计算,包括的运算符有可以连续运算的单目运算符、赋值运算符、条件运算符。
运算符优先级由高到低排列:后缀$\rightarrow$一元 $\rightarrow$
乘除$\rightarrow$
加减$\rightarrow$
移位$\rightarrow$ 
关系$\rightarrow$
相等$\rightarrow$
位与$\rightarrow$
位异或$\rightarrow$
位或$\rightarrow$
逻辑与$\rightarrow$
逻辑或$\rightarrow$
条件$\rightarrow$
赋值$\rightarrow$
逗号
\subsubsection{语句}
在C语言中,语句分为说明语句、表达式语句、控制语句、标签语句、复合语句和块语句。\textbf{在我们所定义的SysY语言中,我们将定义表达式语句和控制语句,其中控制语句分为分支语句、循环语句和转向语句。}
\paragraph{表达式语句} 任意有效表达式都可以作为表达式语句,其形式为表达式后面加上“;”。
\paragraph{分支语句} if语句和if……else语句,由关键字if和else组成。其基本形式如下
\begin{lstlisting}[language = c++]
if(expr){
  stmts
}
else{
  stmts
}
\end{lstlisting}
\paragraph{循环语句} 又称重复语句,用于重复执行某些语句。本实验的循环语句由while语句实现,基本形式如下
\begin{lstlisting}[language = c++]
while{
  stmts
}
\end{lstlisting}
\paragraph{转向语句} 用于从循环体跳出的break语句;用于立即结束本次循环而去继续下一次循环的continue语句;用于立即从某个函数中返回到调用该函数位置的return语句。




%----------------------------------------------
% 语言特性开始




\subsubsection{循环}
\paragraph{循环作用}
有的时候,我们可能需要多次执行同一块代码。一般情况下,语句是按顺序执行的:函数中的第一个语句先执行,接着是第二个语句,依此类推。编程语言提供了更为复杂执行路径的多种控制结构。循环语句允许我们多次执行一个语句或语句组。


\paragraph{循环类型}
在我们的SysY语言中,我们将提供以下几种循环类型,如表\ref{fig:2}所示:
\newpage
\begin{table}[!htbp]
  \centering
  \begin{tabular}{ccccccccccc}
  \toprule  
  循环类型& 描述&\\
  \midrule
  while循环& 当给定条件为真时,重复语句或语句组。它会在执行循环主体之前测试条件。&\\
  for循环& 多次执行一个语句序列,简化管理循环变量的代码。&\\
  do...while 循环& 除了它是在循环主体结尾测试条件外,其他与 while 语句类似。&\\ 
  嵌套循环& 可以在 while、for 或 do..while 循环内使用一个或多个循环。&\\
  \bottomrule
  \end{tabular}
  \caption{循环类型描述}
  \label{fig:2}
\end{table}

每个循环的语法如下:


\begin{lstlisting}[title = while循环, language = c++]
while(condition)
{
   statement(s);
}
\end{lstlisting}


\begin{lstlisting}[title = for循环, language = c++]
for ( init; condition; increment )
{
   statement(s);
}
\end{lstlisting}

\begin{lstlisting}[title = do...while 循环, language = c++]
do
{
   statement(s);

}while( condition );
\end{lstlisting}


\begin{lstlisting}[title = 嵌套循环, language = c++]
for (initialization; condition; increment/decrement)
{
    statement(s);
    for (initialization; condition; increment/decrement)
    {
        statement(s);
        ... ... ...
    }
    ... ... ...
}
\end{lstlisting}

\paragraph{循环控制语句}
在循环中,有一些特殊的关键词可以改变代码的执行顺序,通过它可以实现代码的跳转。我们将提供以下几种循环控制语句,如表\ref{fig:3}所示:
\newpage
\begin{table}[!htbp]
  \centering
  \begin{tabular}{ccccccccccc}
  \toprule  
  控制语句& 描述&\\
  \midrule
  break 语句& 终止循环或 switch 语句,程序流将继续执行紧接着循环或 switch 的下一条语句。&\\
  continue 语句& 告诉一个循环体立刻停止本次循环迭代,重新开始下次循环迭代。&\\
  \bottomrule
  \end{tabular}
  \caption{循环控制语句}
  \label{fig:3}
\end{table}





\subsubsection{函数}

\paragraph{函数的定义}
函数是一组一起执行一个任务的语句。程序中功能相同,结构相似的代码段可以用函数进行描述。函数的功能相对独立,用来解决某个问题,具有明显的入口和出口。函数也可以称为方法、子例程或程序等等。
\paragraph{函数说明}
C语言中,函数必须先说明后调用。函数的说明方式有两种,一种是函数原型,相当于“说明语句”,必须出现在调用函数之前;一种是函数定义,相当于“说明语句+初始化”,可以出现在程序的任何合适的地方。在函数声明中,参数的名称并不重要,只有参数的类型是必需的。函数的声明形式如下所示:
\begin{lstlisting}[language = c++]
return_type function_name(parameter list);
\end{lstlisting}
\paragraph{形式化定义}
C语言中,函数的形式化定义如下所示:
\begin{lstlisting}[language = c++]
return_type function_name(parameter list)
{
   body of the function
}
\end{lstlisting}
\textbf{本实验定义的SysY语言的函数定义完全与此相同。}
\paragraph{函数的参数}
函数可以分为有参函数和无参函数。如果函数要使用参数,则必须声明接受参数值的变量,这些变量称为函数的形式参数。形式参数和函数中的局部变量一样,在函数创建时被赋予地址,在函数退出时被销毁。\\
\\
函数参数的调用分为传值调用和引用调用两种:
\begin{itemize}
    \item 传值调用是将实际的变量的值复制给形式参数,形式参数在函数体中的改变不会影响实际变量
    \item 引用调用是将形式参数作为指针调用指向实际变量的地址,当对在函数体中对形式参数的指向操作时,就相当于对实际参数本身进行的操作。
\end{itemize}

除此之外,函数还可以分为内联函数、外部函数等等,并还可以进行重载等操作。\textbf{本次实验的SysY语言,我们只对函数最基本的功能进行实现。}

\subsubsection{数组}
\paragraph{数组介绍}
在C语言中支持数组数据结构,它可以存储一个固定大小的相同类型元素的顺序集合。数组是用来存储一系列数据,但它往往被认为是一系列相同类型的变量。所有的数组都是由连续的内存位置组成。最低的地址对应第一个元素,最高的地址对应最后一个元素。

\paragraph{数组声明}
在我们所定义的SysY语言中要声明一个数组,需要指定元素的类型和元素的数量,如下所示:
\begin{lstlisting}[language = c++]
type arrayName [ arraySize ];
\end{lstlisting}
arraySize 必须是一个大于零的整数常量,type 可以是任意有效的 C 数据类型。


\paragraph{数组初始化}
对于数组的初始化,可以使用初始化语句,如下所示:
\begin{lstlisting}[language = c++]
double balance[5] = {1000.0, 2.0, 3.4, 7.0, 50.0};
\end{lstlisting}

\paragraph{访问数组元素}
数组元素可以通过数组名称加索引进行访问。元素的索引是放在方括号内,跟在数组名称的后边。例如:
\begin{lstlisting}[language = c++]
double salary = balance[9];
\end{lstlisting}


\subsubsection{指针}
\paragraph{指针概念}
在我们定义的SysY语言中,和C语言一样,指针指的是内存地址,指针变量是用来存放内存地址的变量。就像其他变量或常量一样,必须在使用指针存储其他变量地址之前,对其进行声明。指针变量声明的一般形式为:
\begin{lstlisting}[language = c++]
type *var_name;
\end{lstlisting}
type 是指针的基类型,它必须是一个有效的数据类型,var\_name 是指针变量的名称。用来声明指针的星号 * 与乘法中使用的星号是相同的。所有实际数据类型,不管是整型、浮点型、字符型,还是其他的数据类型,对应指针的值的类型都是一样的,都是一个代表内存地址的长的十六进制数。不同数据类型的指针之间唯一的不同是,指针所指向的变量或常量的数据类型不同。
\paragraph{指针用法}
使用指针时会频繁进行以下几个操作:定义一个指针变量、把变量地址赋值给指针、访问指针变量中可用地址的值。这些是通过使用一元运算符 * 来返回位于操作数所指定地址的变量的值。



\subsubsection{字符串}
和 C 语言中的定义类似,我们的SysY语言的字符串实际上是使用空字符 $\textbackslash0$ 结尾的一维字符数组。因此,$\textbackslash0$ 是用于标记字符串的结束。

\textbf{空字符(Null character)}又称结束符,缩写 $NUL$,是一个数值为 0 的控制字符,$\textbackslash0$ 是转义字符,意思是告诉编译器,这不是字符 0,而是空字符。并且在我们的编译器中并不需要把 null 字符放在字符串常量的末尾,此编译器会自动在初始化数组时,把 $\textbackslash0$ 放在字符串的末尾。


% 语言特性结束
%----------------------------------------------











\subsection{形式化定义}
接下来,我们将采用CFG即上下文无关文法对SysY语言进行形式化定义。上下文无关文由一个终结符号集合$V_T$、一个非终结符号集合$V_N$、一个产生式集合$P$和一个开始符号$S$四个元素组成。在接下来的定义中,数位、符号和黑体字符串将被看作终结符号,斜体字符串将被看作非终结符号。若多个产生式以一个非终结符号为头部,则这些产生式的右部可以放在一起,并用|分割。
\begin{table}[!htbp]
  \centering
  \begin{tabular}{ccccccccccc}
  \toprule  
  名称& 符号& 名称& 符号&\\
  \midrule
  声明语句& $decl$& 标识符& $\tt{id}$\\
  标识符列表& $idlist$&数据类型& $type$\\
  表达式&$expr$&一元表达式&$unary\_expr$\\
  赋值表达式&$assign\_expr$&逻辑表达式&$logical\_expr$\\
  算数表达式&$math\_expr$&关系表达式&$relation\_expr$\\
  数字&$\tt{digit}$&整数&$decimal$\\
  符号和字母&$\tt{character}$&常量定义&$const\_init$\\
  分配内存& $allocate$& 回收内存& $recovery$\\
  语句& $stmt$& 循环语句& $loop\_stmt$\\
  分支语句& $selection\_stmt$& 跳转语句& $jmp\_stmt$\\
  函数定义& $funcdef$& 函数参数& $para$\\
  函数参数列表& $paralist$& 函数名称& $funcname$\\
  函数返回值& $re\_type$\\
  \bottomrule
  \end{tabular}
  \caption{下文中各符号含义}
\end{table}


\subsubsection{变量声明}
变量可以声明分为仅声明变量和声明变量且赋初值。(整型变量只支持十进制)数组由指针实现。
\begin{align*}
idlist \rightarrow\ & idlist,\tt{id}\ |\ \tt{id}\\
type \rightarrow\ & \tt{int} \ | \ \tt{char}\\
decl \rightarrow\ & type\ idlist \ | \\
&type\ \text{$\tt{id}$} = logical\_expr \ |\\
&type\ \text{$\tt{id}$} = unary\_expr \ | \\
&type \ *\text{$\tt{id}$} = \tt{\&id}
\end{align*}
\subsubsection{常量声明}
常量包括整型常量(十进制)和字符型常量。
\begin{align*}
const\_init\rightarrow\ \tt{const}\ \it{type}\ \tt{id} = \it{unary\_expr}
\end{align*}
\subsubsection{表达式}
表达式可以分为一元表达式、赋值表达式、逻辑表达式、算数表达式、关系表达式。
$$
expr \rightarrow \ unary\_expr \ |\ assign\_expr \ |\ logical\_expr \ |\ math\_expr \ |\ relation\_expr
$$
\subsubsection{赋值表达式}
赋值表达式不能对常量进行赋值。
\begin{align*}
\tt{digit}\rightarrow\  &\tt{0}\ | \ \tt{1}\ | \ \tt{2}\ | \ \tt{3}\ | \ \tt{4}\ | \ \tt{5}\ | \ \tt{6}\ | \ \tt{7}\ | \ \tt{8}\ | \ \tt{9}\\
decimal \rightarrow\  &\tt{digit}\ | \ \text{$decimal$}\ \tt{digit}\\
\tt{character} \rightarrow \ &\tt{\_ \ |\ a-z \ |\ A-Z}\\ 
unary\_expr \rightarrow \ &decimal \ |\ '\tt{character}' \ |\ \tt{id}\\
assign\_expr \rightarrow \ &\text{$\tt{id}$} = unary\_expr \ | \\ 
&\text{$\tt{id}$} = logical\_expr\ | \\
&\text{$\tt{id}$} = funcnmae(paralist)
\end{align*}
\subsubsection{逻辑表达式}
逻辑表达式包括逻辑与、逻辑或、逻辑非运算。
\begin{align*}
logical\_expr \rightarrow\ &unary\_expr\ |\\
&!(logical\_expr)\ |\\
&logical\_expr || logical\_expr\ |\\
&logical\_expr \&\& logical\_expr
\end{align*}
\subsubsection{关系表达式}
关系表达式包括判断两个值是否相等或比较两值的大小。
\begin{align*}
relation\_expr \rightarrow \ &unary\_expr==unary\_expr\ |\\
&unary\_expr\ != unary\_expr\ |\\
&unary\_expr > unary\_expr\ |\\
&unary\_expr < unary\_expr\ |\\
&unary\_expr >= unary\_expr\ |\\
&unary\_expr <= unary\_expr\\
\end{align*}
\subsubsection{算数表达式}
算数表达式包括加、减、乘、除、取模、取负六种运算。
\begin{align*}
math\_expr \rightarrow \ &unary\_expr\ |\\
&-unary\_expr\ |\\
&math\_expr + math\_expr\ |\\
&math\_expr - math\_expr\ |\\
&math\_expr * math\_expr\ |\\
&math\_expr\ / \ math\_expr\ |\\
&math\_expr\ \% \ math\_expr\\
\end{align*}
\subsubsection{系统操作}
系统操作包括分配内存和回收内存。
\begin{align*}
allocate \rightarrow \ &type\ \tt{*id} = \text{$\tt{new} $}\ \it{type}\ [\text{$\tt{id}$}]\ |\\
&type\ \tt{*id} =  \tt{new}\ \it{type}\ [decimal]\\
recovery \rightarrow \ &\tt{delete}\ [] \ id
\end{align*}
\subsubsection{语句}
语句包括循环语句、分支语句、跳转语句。
\begin{align*}
stmt\rightarrow\ &loop\_stmt \ |\ selecion\_stmt \ |\ jmp\_stmt \ |\\ 
&expr;\ stmt\ | \\
&expr\ 
\end{align*}
\subsubsection{循环语句}
循环语句利用while实现。
$$
loop\_stmt \rightarrow \ \tt{while}\it{(expr)}\ \it{\{stmt\}}
$$
\subsubsection{分支语句}
分支语句利用if else实现。
\begin{align*}
selection\_stmt \rightarrow &\tt{if}(\it{expr})\ \{stmt\} \ | \\
&\tt{if}(\it{expr})\ \{stmt\}\ \tt{else} \ \it{\{stmt\}}
\end{align*}
\subsubsection{跳转语句}
跳转语句包括继续执行循环、退出循环和返回值。
\begin{align*}
jmp\_stmt \rightarrow\ &\tt{continue}\ |\\
&\tt{break} \ |\\
&\tt{return} \ |\\
&\tt{return}\ \it{expr}
\end{align*}
\subsubsection{函数} 
函数的返回值有整型、字符型、指针型三种,参数有整型、字符型、指针型、引用四种。(数组由指针实现)
\begin{align*}
funcdef \rightarrow\ &re\_type\ \tt{funcname}\it{(paralist)\ stmt}\\
paralist \rightarrow\ & para, paralist \ |\ para\\
para \rightarrow\ &type\ \tt{id}\ | \\
&type* \ \tt{id}\ | \\
&type\& \ \tt{id}\ \\
re\_type \rightarrow \ &type\ |\ type*\ |\ \tt{void}
\end{align*}
\subsubsection{注释}
本实验定义的SysY语言将使用//作为注释开始的符号,/*\ */将不被使用。在编译的预处理阶段,//后到行末的内容将被删除。
\subsubsection{输入输出流}
输入输出流将由SysY语言提供的I/O函数实现,在此不再定义。
%----------------------------------------------------------------
\newpage




\section{汇编编程}
本人在Linux ubuntu 4.15.0-142-generic环境下,利用vim将C++文件手动翻译成汇编文件。通过\fbox{arm-linux-gnueabihf-g++ 汇编文件名.S -o 目标文件名 -static}指令生成可执行程序,再通过\fbox{qemu-arm ./目标文件名}指令进行运行调试,最终对C++文件到arm源文件的转换有了一定的了解。手动编写的汇编文件已经上传于gitlab,链接如下:\\
\href{git@gitlab.eduxiji.net:nku2021-anthony/compilers.git}{SSH: git@gitlab.eduxiji.net:nku2021-anthony/compilers.git}\\
\href{https://gitlab.eduxiji.net/nku2021-anthony/compilers.git}{HTTP:  https://gitlab.eduxiji.net/nku2021-anthony/compilers.git}
\subsection{一些基本操作}
\subsubsection{关于函数的栈指针操作}
在现如今的计算机体系结构中,栈是向下增长的,即由大端地址增长到小端地址。在arm汇编中,fp寄存器用作帧指针,sp寄存器指向栈顶,在跳转语句调用子函数时会将当前的PC保存在lr寄存器中。当调用一个函数时,该函数首先将 fp 的当前值保存在堆栈上。然后,它将 sp 寄存器的值保存在 fp 寄存器中。然后递减 sp 寄存器来为本地变量分配空间。fp 寄存器用于访问本地变量和参数,局部变量位于帧指针的负偏移量处,传递给函数的参数位于帧指针的正偏移量。
当函数返回时,fp 寄存器被复制到 sp 寄存器中,这将释放用于局部变量的堆栈,函数调用者的 fp 寄存器的值由pop从堆栈中恢复。\cite{zhn}
\begin{lstlisting}[title = 函数被调用时的基本框架]
push {fp, lr} @将fp的当前值保存在堆栈上,然后将sp寄存器的值保存在fp中,lr中存储的是pc的保存在lr中
@被调用的函数体
pop {fp, lr} 
bx lr @相当于mov pc, lr   恢复上下文
\end{lstlisting}
\subsubsection{I/O操作}
已知在arm汇编架构中,关于C++文件输入输出流cin、cout没有直接相对应的指令,所以我们将会调用从C语言中继承的scanf和printf在其中使用。已知,在调用这两个函数时,r0寄存器保存的是自定义字符串的地址,r1寄存器保存的是“\%d”中将要被替换的内容,具体代码如下:
\begin{lstlisting}[title = arm汇编语言中的输入流模版]
sub sp, sp, #4 @在栈中开辟一块大小为4的内存地址,用于存储即将输入的数据
ldr r0, =_cin
mov r1, sp @将sp的值传输给r1寄存器,使scanf传入的值存储在栈上,即栈顶的值是输入的值
bl scanf
ldr 任意寄存器, [sp, #0] @取出sp指针指向的地址中的内容,即栈顶中的内容(输入的值)
add sp, sp, #4 @恢复栈顶,释放内存空间

.data @数据段
_cin 
  .asciz "%d"
\end{lstlisting}
\newpage
\begin{lstlisting}[title = arm汇编语言中的输出流模版]

ldr r0, =_bridge
mov r1, 要输出内容所在寄存器 
bl printf

.data
_bridge:
  .asciz "%d\n"
\end{lstlisting}
\subsection{汇编程序编写}
\subsubsection{斐波那契数列}
\begin{lstlisting}[title = 源程序, language = c++]
#include <iostream>
using namespace std;
int main()
{
    int a, b, i, t, n;
    a = 0;
    b = 1;
    i = 1;
    cin >> n;
    cout << "a:" << a << endl;
    cout << "b:" << b << endl;
    cout << "we are going to loop now! " << endl;
    while (i < n)
    {
        t = b;
        b = a + b;
        cout << b << endl;
        a = t;
        i = i + 1;
    }
    return 0;
}
\end{lstlisting}
\begin{lstlisting}[title = 改写后的汇编代码]
  .arch armv7-a @处理器架构
  .arm
@r0是格式化字符串,r1是对应的printf对应的第二个参数
@代码段
@主函数
  .text @代码段
  .global main
  .type main, %function
main: 
  push {fp, lr} @将fp的当前值保存在堆栈上,然后将sp寄存器的值保存在fp中,lr中存储的是pc的保存在lr中
  sub sp, sp, #4 @在栈中开辟一块大小为4的内存地址,用于存储即将输入的数据
  ldr r0, =_cin
  mov r1, sp @将sp的值传输给r1寄存器,使scanf传入的值存储在栈上,即栈顶的值是n
  bl scanf
  ldr r6, [sp, #0] @取出sp指针指向的地址中的内容,即栈顶中的内容(输入的n的值)
  add sp, sp, #4 @恢复栈顶,释放内存空间

  @测试是否写入
  @ldr r0, =_bridge3
        @mov r1, r2
        @bl printf

  mov r4, #0 @a = 0
  mov r5, #1 @b = 1
  mov r7, #1 @i = 1
  @r4中存a的值,r5中存b的值,r7中存i的值,r6中存n的值
  ldr r0, =_bridge
  mov r1, r4 @将r4中的值即a的值赋予r1
  bl printf @打印a的值
  ldr r0, =_bridge2
  mov r1, r5 @将r5中的值即b的值赋予r1
  bl printf @打印b的值
        ldr r0, =_bridge4
        bl printf
  
  @输出进行调试
  @ldr r0, =_bridge3
        @mov r1, r6
        @bl printf
        @ldr r0, =_bridge3
        @mov r1, r7
        @bl printf

Loop:
  @输出进行调试
  @ldr r0, =_bridge4
  @bl printf
  @ldr r0, =_bridge3
  @mov r1, r6
        @bl printf
  @ldr r0, =_bridge3
        @mov r1, r7
        @bl printf

  cmp r6, r7
  ble RETURN @比较r7和r6(即i和n)的大小用于跳转
  mov r8, r5 @t = b @r8为临时变量的寄存器
  add r5, r5, r4 @b = a + b
  ldr r0, =_bridge3
  mov r1, r5 @将r5中的值即b的值赋予r1
  bl printf  @cout << b << endl;
  mov r4, r8 @a = t
  add r7, r7, #1 @i = i + 1
  b Loop

RETURN:
  pop {fp, lr} @上下文切换
  bx lr @return 0
.data @数据段
_cin:
  .asciz "%d"

_bridge:
  .asciz "a:%d\n"

_bridge2:
  .asciz "b:%d\n"

_bridge3:
        .asciz "%d\n"

_bridge4:
  .asciz "We are going to loop now! \n"

.section .note.GNU-stack,"",%progbits @ do you know what's the use of this :-)
\end{lstlisting}
运行后的结果如图\ref{fig:1}所示。可以看出,手写汇编代码正确。
\begin{figure}[H]
    \centering
    \includegraphics[scale=0.55]{NKU.png}
    \caption{斐波那契程序运行结果}
    \label{fig:1}
\end{figure}
\subsubsection{求平方}
\begin{lstlisting}[language = c++, title = 源程序]
#include <iostream>
using namespace std;
int square(int a){
        int m;
        m = a * a;
        return m;
}
int main()
{
        int a, s_a;
        cin >> a;
        s_a = square(a);
        cout << s_a << endl;
        return 0;
}
\end{lstlisting}
\begin{lstlisting}[title = 改写后的汇编代码]
.arch armv7-a
  .arm

  .text @代码段
  .global square
square: @function int square(int a)
  str fp, [sp, #-4]! @pre-index mode, sp = sp -4, push fp
  mov fp, sp
  sub sp, sp, #8 @为本地变量开辟空间
  str r0, [fp, #-8] @r0 = [fp, #-8] = a
  mul r1, r0, r0
  mov r0, r1
  add sp, fp, #0
  ldr fp, [sp], #4
  bx lr

  .text @代码段
  .global main
  .type main, %function
main:
  push {fp, lr}
  sub sp, sp, #4
  ldr r0, =_cin
  mov r1, sp
  bl scanf
  ldr r0, [sp, #0] @取出输入的内容放入r0中
  add sp, sp, #4
  bl square
  mov r1, r0
  ldr r0, =_cout
  bl printf
  mov r0, #0
  pop {fp, lr}
  bx lr

.data @数据段
_cin:
  .asciz "%d"

_cout:
  .asciz "%d\n"

.section .note.GNU-stack,"",%progbits @ do you know what's the use of this :-)

\end{lstlisting}
运行后的结果如图\ref{fig:2}所示。可以看出,手写汇编代码正确。
\begin{figure}[H]
    \centering
    \includegraphics[scale=0.5]{NKU.png}
    \caption{求平方程序运行结果}
    \label{fig:2}
\end{figure}








%----------------------------------------------
% 汇编编码开始

\subsubsection{斐波那契数列}

\begin{lstlisting}[title = 斐波那契数列源程序, language = c++]
#include <iostream>
using namespace std;
int main()
{
    int a, b, i, t, n;
    a = 0;
    b = 1;
    i = 1;
    cin >> n;
    cout << a << endl;
    cout << b << endl;
    cout << "this is zyl's loop" << endl;
    while (i < n)
    {
        t = b;
        b = a + b;
        cout << b << endl;
        a = t;
        i = i + 1;
    }
    return 0;
}
\end{lstlisting}


\begin{lstlisting}[title = 依照斐波那契数列手写的汇编代码]
  .arch armv7-a @处理器架构
  .arm
@r0是格式化字符串,r1是对应的printf对应的第二个参数
  .text @代码段
  .global main
  .type main, %function
@主函数
main: 
  push {fp, lr} @将fp的当前值保存在堆栈上,然后将sp寄存器的值保存在fp中,lr中存储的是pc的保存在lr中
  @ 此处是获取输入的数据
  sub sp, sp, #4 @在栈中开辟一块大小为4的内存地址,用于存储即将输入的数据
  ldr r0, =_cin
  mov r1, sp @将sp的值传输给r1寄存器,使scanf传入的值存储在栈上,即栈顶的值是n
  bl scanf @执行输入操作
  ldr r6, [sp, #0] @取出sp指针指向的地址中的内容,即栈顶中的内容(输入的n的值)
  add sp, sp, #4 @恢复栈顶,释放内存空间
  @此处是对变量的赋值操作
  mov r4, #0 @a = 0
  mov r5, #1 @b = 1
  mov r7, #1 @i = 1
  @r4中存a的值,r5中存b的值,r7中存i的值,r6中存n的值
  ldr r0, =_bridge1
  mov r1, r4 @将r4中的值即a的值赋予r1
  bl printf @cout << a << endl;
  ldr r0, =_bridge1
  mov r1, r5 @将r5中的值即b的值赋予r1
  bl printf @cout << b << endl;
  ldr r0, =_bridge2
  bl printf @输出自己加载的字符串,用来证实原创性
@此处是while循环
Loop:
  cmp r6, r7
  ble RETURN @比较r7和r6(即i和n)的大小用于跳转
  mov r8, r5 @t = b @r8为临时变量的寄存器
  add r5, r5, r4 @b = a + b
  ldr r0, =_bridge1
  mov r1, r5 @将r5中的值即b的值赋予r1
  bl printf  @cout << b << endl;
  mov r4, r8 @a = t
  add r7, r7, #1 @i = i + 1
  b Loop

RETURN:
  pop {fp, lr} @上下文切换
  bx lr @return 0
.data @数据段
_cin:
  .asciz "%d"

_bridge1:
  .asciz "%d\n"

_bridge2:
  .asciz "this is zyl's loop\n"

.section .note.GNU-stack,"",%progbits @ do you know what's the use of this :-)
\end{lstlisting}



运行后的结果如图\ref{fig:1}所示。可以看出,手写汇编代码正确。
\begin{figure}[H]
    \centering
    \includegraphics[scale=0.4]{1.png}
    \caption{斐波那契程序运行结果}
    \label{fig:1}
\end{figure}








\subsubsection{自定义程序}
本人在汇编程序中主要撰写的语言特性包括if 分支语句,以及 while/for 循环,支持算术运算(加减乘除、按位与或等)、逻辑运算(逻辑与或等)、关系运算(不等、等于、大于、小于等),由于在斐波那契数列中有一点语言特性并没有体现出来(比如逻辑运算和if 分支语句),所以又撰写了一个较简单的自定义程序来作为对斐波那契程序中缺少的语言特性的补充,自定义程序如下:

\begin{lstlisting}[title = 自定义源程序, language = c++]
#include <iostream>
using namespace std;

int main()
{
    int a;
    cin >> a;
    if (!(a > 100))
        cout << "your number is below 100!\n";
    else
        cout << "your number is greater than 100!\n";
    return 0;
}
\end{lstlisting}


\begin{lstlisting}[title = 自定义程序改写后的汇编代码]
  .arch armv7-a @处理器架构
  .arm
@r0是格式化字符串,r1是对应的printf对应的第二个参数
  .text @代码段
  .global main
  .type main, %function
@主函数
main: 
  push {fp, lr} @将fp的当前值保存在堆栈上,然后将sp寄存器的值保存在fp中,lr中存储的是pc的保存在lr中
  @此处是赋值操作
  sub sp, sp, #4 @在栈中开辟一块大小为4的内存地址,用于存储即将输入的数据
  ldr r0, =_cin
  mov r1, sp @将sp的值传输给r1寄存器,使scanf传入的值存储在栈上,即栈顶的值是a
  bl scanf
  ldr r6, [sp, #0] @取出sp指针指向的地址中的内容,即栈顶中的内容(输入的a的值)
  add sp, sp, #4 @恢复栈顶,释放内存空间

  @r6中存a的值
  @接下来是判断操作
  cmp r6, #0x64
  ble COMPARE @比较0x64和r6(即100和a)的大小用于跳转
  @输出"your number is greater than 100!\n"
  ldr r0, =_bridge3
  bl printf
  b RETURN @不论怎么样都跳转

COMPARE:
  @输出"your number is below 100!\n"
  ldr r0, =_bridge2
  bl printf

RETURN:
  pop {fp, lr} @上下文切换
  bx lr @return 0
.data @数据段
_cin:
  .asciz "%d"

_bridge1:
  .asciz "%d\n"

_bridge2:
  .asciz "your number is below 100!\n"

_bridge3:
  .asciz "your number is greater than 100!\n"

.section .note.GNU-stack,"",%progbits @ do you know what's the use of this :-)
\end{lstlisting}



运行后的结果如图\ref{fig:2}所示。可以看出,手写汇编代码正确。
\begin{figure}[H]
    \centering
    \includegraphics[scale=0.4]{2.png}
    \caption{自定义程序运行结果}
    \label{fig:2}
\end{figure}



% 汇编编码结束
%----------------------------------------------













% \newpage
% \section{概述}
% %——————————————————————————————————————
% \subsection{第一节}
% 如图\ref{fig:1}所示
% \begin{figure}[H]
%     \centering
%     \includegraphics[scale=0.3]{NKU.png}
%     \caption{Caption}
%     \label{fig:1}
% \end{figure}

% 表
% \begin{table}[!htbp]
%   \centering
%   \begin{tabular}{ccccccccccc}
%   \toprule  
%   N/n$\backslash$Algo& naive-conv& naive-pool& omp-conv& omp-pool\\
%   \midrule
%   64/2& 0.0167& 0.01255& 0.04142& 0.03799\\
%   64/4& 0.03599&0.0394& 0.0458& 0.0421\\
%   \bottomrule
%   \end{tabular}
%   \caption{性能测试结果(4线程)(单位:ms)}
% \end{table}

% 带单元格表格
% \begin{table}[!htbp]
%   \centering
%   \begin{tabular}{|c|c|c|c|c|c|c|}
%   \hline
%   \multicolumn{2}{|c|}{ \multirow{2}*{$Cost$} }& \multicolumn{5}{c|}{To}\\
%   \cline{3-7}
%   \multicolumn{2}{|c|}{}&$A$&$B$&$C$&$D$&$E$\\
%   \hline
%   \multirow{3}*{From}&$B$&7&0&1&3&8\\
%   \cline{2-7}
%   &$C$&8&1&0&2&7\\
%   \cline{2-7}
%   &$D$&8&3&2&0&5\\
%   \hline
%   \end{tabular}
%   \caption{结点C距离向量表(无毒性逆转)}
% \end{table}

% %——————————————————————————————————————
% \subsection{第二节}
% 伪代码

% \begin{breakablealgorithm} 
%   \caption{初始化obj文件信息——对应MeshSimplify类中readfile函数,Face类calMatrix函数} 
%   \begin{algorithmic}[1] %每行显示行号  
%       \Require obj文件,顶点、边、面列表
%       \Ensure 是否读取成功
%       \Function {calMatrix}{$Face$}  
%               \State $normal \gets e1×e2$  
%               \State $normal \gets normal/normal.length$
%               \State $temp[] \gets {normal.x, normal.y, normal.z, normal· Face.v1}$
%               \State $Matrix[i][j]=temp[i] * temp[j]$ 
%               \State \Return{$Matrix$}  
%       \EndFunction
%       \State 根据obj的v和f区分点面信息,读取并加入列表
%       \State $scale \gets $记录点坐标中距离原点最远的分量,以便后续OpenGL进行显示
%       \State $ori \gets $记录中心点,便于OpenGL显示在中心位置,避免有的obj偏移原点较多
%       \State 根据三角面片信息,计算一个面的三条边
%       \State 计算每个面的矩阵$\gets calMatrix$
%       \State 将每个面的矩阵加到各点,由点维护\\
%       \Return True
%   \end{algorithmic}  
% \end{breakablealgorithm}

% 代码
% \begin{lstlisting}[title=逐列访问平凡算法,frame=trbl,language={C++}]
%   void ord()   
%   {
%       double head,tail,freq,head1,tail1,timess=0; // timers
%       init(N);
%       QueryPerformanceFrequency((LARGE_INTEGER *)&freq );
%       QueryPerformanceCounter((LARGE_INTEGER *)&head);
%       for (int i=0; i<NN; i++)
%           for (int j=0; j<NN; j++)
%               col_sum[i] += (b[j][i]*a[j]);
%       QueryPerformanceCounter ((LARGE_INTEGER *)& tail) ;
%       cout << "\nordCol:" <<(tail-head)*1000.0 / freq<< "ms" << endl;
%   }
% \end{lstlisting}


% %——————————————————————————————————————
% \subsection{第三节}

% 参考文献\cite{adams1995hitchhiker}\cite{shin2016deep}
    
% 多行公式
% \begin{align}
%   a+b = a + b \\
%   \frac{a+b}{a-b}
% \end{align}

% 行内公式:$\sum^N_{i=1}$

% \textbf{超链接}  \href{http://youtube.com/}{YouTube}

% 带标号枚举
% \begin{enumerate}
%   \item 1
%   \item 2
% \end{enumerate}

% 不带标号枚举
% \begin{itemize}
%   \item 1
%   \item 2
% \end{itemize}

% \xiaosi{切换字体大小}

%----------------------------------------------------------------
\newpage





\section{总结与展望}
\subsection{实验总结}
本次实验是编译原理的第二次实验,首先对SysY的语言特性更加的了解,接下来对源代码进行分析并手写相应的汇编代码,最后成功也在arm汇编器下调试通过。
\subsection{未来展望}
本次是第二次实验,对编译方面的相关知识有了更深入、更全面的理解,对于各个部分所要实现的功能以及未来自己所要实现的语言特性有了一定的掌握,期望我们小组在这个学期可以取得一个满意的结果。
%----------------------------------------------------------------
% \newpage
% \bibliographystyle{plain}
% \bibliography{references} 
%---------------------------------------------------------------
\newpage
% \vspace*{300pt}
\bibliographystyle{plain}
\bibliography{references} 
\end{document}
